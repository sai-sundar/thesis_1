% Chapter Template

\chapter*{Introduction} % Main chapter title
%\addcontentsline{toc}{chapter}{Introduction}
\addcontentsline{toc}{chapter}{\protect\numberline{}Introduction}
\label{Introduction} % Change X to a consecutive number; for referencing this chapter elsewhere, use \ref{ChapterX}

%----------------------------------------------------------------------------------------
%	SECTION 1
%----------------------------------------------------------------------------------------

\section{Motivation}

The use of deep learning in medical imaging has been on the rise over the last few years. It has widely been used in various tasks across medical imaging such as image segmentation  (\cite{ronneberger2015u,guo2019deep,sinha2019multi,dolz2018hyperdense,hatt2018first}), image denoising (\cite{kadimesetty2018convolutional,li2020sacnn,chen2017low,yang2018low}), image analysis (\cite{litjens2017survey,amyar20193,cui2018artificial}). %\noteAB{Add more references}. 
Deep learning based algorithms produce faster results along with best possible quality in accordance with existing state of the art methods (\cite{leuschner2021quantitative}). Medical Image reconstruction too has benefited hugely with the advancement of deep learning (\cite{reader2020deep,zhang2020review}).
Medical Image reconstruction corresponds to the task of mapping raw projection data retrieved from the detector to image domain data. During the course of this thesis, the focus has been towards \ac{PET} and \ac{CT} image reconstruction. Both these modalities present a unique of set of challenges for image reconstruction. 
 
\ac{PET} imaging is a form of emission tomography wherein the image reconstruction task revolves around identifying the radio-tracer distribution emitted from the patient. A \ac{PET} image gives functional information about the organs in a patient making it invaluable for oncology. Some of the challenges in \ac{PET} image reconstruction are scatter, attenuation and difficulty in identifying the exact annihilation point of the electron-positron. Despite being the most sensitive emission tomography modality, the number of photons captured is low relative to the photons emitted contributing to further image degradation. These challenges result in very noisy images when reconstructed with analytical algorithms. These challenges are addressed by  Iterative/Model-based approaches which take into account detector geometry, noise statistics and approximate scatter and attenuation correction resulting in better image quality. 

\ac{CT} imaging on the other hand is an example of transmission tomography. The extent of attenuation undergone by X-Rays that pass through a patient are measured to obtain attenuation maps. In \ac{CT} imaging research, there has been active interest in sparse-view and low-dose reconstruction scenarios. In both cases, severe artifacts are introduced in  reconstructed images either due to incomplete projections or low counts. Many established model-based iterative methods account for the low-dose and sparse-view settings to remove artifacts and noise from the reconstruction (\cite{nuyts1998iterative,Elbakri2002,liu2013total}). However, these methods for require the knowledge of the noise and artifacts statistics and generally have longer reconstruction times (\cite{kim2014combining}). 


The main tasks involved in image reconstruction can be broadly categorized into three: sinogram correction, domain translation from sinogram to image, and image correction. Algorithms either tackle the three task individually or simultaneously account for them. One can relate to these tasks in the domain of Computer Vision wherein deep learning architectures have revolutionized the field by producing the state of the art results in most applications (\cite{guo2016deep}). For example, effective use of deep learning-based methods is seen in dealing with image denoising (\cite{kadimesetty2018convolutional,li2020sacnn,chen2017low,yang2018low}), super resolution (\cite{ledig2017photo,lim2017enhanced}) and image-to-image translation (\cite{isola2017image,zhu2017unpaired}) tasks. The continuous improvement in the availability of public data has further propelled interest in data-driven medical image reconstruction making it an active area of research. This thesis aims to explore novel deep learning approaches for \ac{PET} and \ac{CT} image reconstruction. Most common ways to introduce deep learning architectures in the image reconstruction pipeline are for pre-processing to correct raw projection data from the detector and post-processing to improve images reconstructed with existing methods. Another way is to embed the network into an iterative algorithm to enable faster convergence. The relatively less explored way called direct image reconstruction is to utilize neural networks alone for the entire reconstruction process. In this thesis \ac{CNN} approaches are proposed for direct image reconstruction with neural networks. 


\section{Thesis Organization}

This thesis is divided into six chapters with the first two chapters being introduction and literature review, followed by three chapters that focus on different deep learning methods explored during the thesis, and finally conclusion and perspectives. In the introduction various aspects of \ac{PET} and \ac{CT} image reconstruction are discussed along with the relevant background in deep learning background. The second chapter throws light on deep learning applied to medical image reconstruction and reviews the state of the art approaches in the scope of this thesis. 
In chapter 3 , we discuss reconstruction framework \ac{DUGAN} for \ac{PET} and \ac{CT} image reconstruction. A novel method for Sparse-view \ac{CT} reconstruction called \ac{LRRCED} is covered in chapter 4. A modified version of \ac{LRRCED} for total body \ac{PET} is discussed in chapter 5. Potential improvements and ideas for future work are presented in the final chapter. 

%----------------------------------------------------------------------------------------
%	SECTION 3
%----------------------------------------------------------------------------------------
